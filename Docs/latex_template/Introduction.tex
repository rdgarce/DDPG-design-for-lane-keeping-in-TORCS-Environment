\chapter*{Introduzione}

La storia ci insegna che l'uomo si è da sempre impegnato per automatizzare e controllare i processi che fanno parte della sua vita. Si pensi al caso dell'aratro antico che permetteva di smuovere il terreno utilizzando il lavoro di un animale piuttosto che quello del contadino, oppure a quello del mulino che, sfruttando vento o acqua, permetteva la macinazione del grano senza necessità di sforzi fisici.

Nell'era dell'Informatica e dei Big Data questo impegno non si è ridotto, anzi è stato traslato anche verso processi che fino a pochi anni fa si ritenevano ad esclusivo appannaggio degli esseri umani.\newline

Se oggi è possibile automatizzare processi come la classificazione di imma-\\gini, convertire testo in codice programmabile o effettuare interpolazione di frame nei video è sopratutto grazie alla grande evoluzione che l'Intelligenza Artificiale sta avendo negli ultimi anni.

Anche in materia di Controlli non mancano esempi di utilizzo delle Neural Network, come nel caso della modellazione del catalizzatore a tre vie in ambito Automotive dove una rete neurale è stata utilizzata per modellare le velocità delle reazioni chimiche che avvengono al suo interno, o del controllo della dinamica non lineare di un missile con una CMAC.\newline

In questa tesi si affronta uno dei principali problemi dell'Autonomous Driving, il lane keeping, ovvero il problema del mantenimento della carreggiata per un'auto a guida autonoma. Il problema viene affrontato nell'ambiente di simulazione di guida TORCS utilizzando una tecnica di Reinforcement Learning.\newline
\clearpage

Il lavoro è strutturato come segue. Nel Capitolo 1 si presenta il Reinforcement Learning e si fa una panoramica del Deep Reinforcement Learning, con una distinzione tra i metodi Value Based, Policy Based e Actor-Critic. Nel Capitolo 2 si definisce il problema del lane keeping nel caso di studio specifico. Nel Capitolo 3 si affronta l'approccio utilizzato, ovvero il design del controllore scelto. Nel Capitolo 4 verranno comparati i risultati ottenuti dalle diverse architetture di controllo sperimentate.